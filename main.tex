%allowed options finnish, swedish, english
%after changing the language you may be forced to use recompile from scratch to get rid of errors
\documentclass[finnish]{tktltiki}
\usepackage{epsfig}
\usepackage{subfigure}
\usepackage{url}
\usepackage[utf8]{inputenc}
\usepackage{mathabx}

% For picture placement
\usepackage{float} 
 
\usepackage{listings}
\usepackage{color}
 
% For code blocks 
\definecolor{codegreen}{rgb}{0,0.6,0}
\definecolor{codegray}{rgb}{0.5,0.5,0.5}
\definecolor{codepurple}{rgb}{0.58,0,0.82}
\definecolor{backcolour}{rgb}{0.95,0.95,0.92}
 
\lstdefinestyle{mystyle}{
    backgroundcolor=\color{backcolour},   
    commentstyle=\color{codegreen},
    keywordstyle=\color{magenta},
    numberstyle=\tiny\color{codegray},
    stringstyle=\color{codepurple},
    basicstyle=\footnotesize,
    breakatwhitespace=false,         
    breaklines=true,                 
    captionpos=b,                    
    keepspaces=true,                 
    numbers=left,                    
    numbersep=5pt,                  
    showspaces=false,                
    showstringspaces=false,
    showtabs=false,                  
    tabsize=4
}
 
\lstset{style=mystyle}


\lstdefinestyle{tree}{
        literate={├}{|}1 {─}{--}1 {└}{+}1 {│}{|}1
    }

\begin{document}

%\doublespacing
%\onehalfspacing
\singlespacing

\title{Euroopan Unionin uusi tietosuoja-asetus ja siihen liittyvä auditointi ja muutostyö TKO-äly ry:lle}
\author{Heikki Ahonen}
\date{\today}

\maketitle

% \classification{\protect{ ONKO TÄMÄ TARPEEN? }}



\keywords{Euroopan Unioni, EU, Yleinen tietosuoja-asetus, GDPR, henkilötiedot, auditointi}

\begin{abstract}

Toukokuussa 2018 voimaan astui Euroopan unionin uusi tietosuoja-asetus. Asetuksen myötä myös Suomen tietosuojalaki uudistettiin vastaamaan uutta asetusta. Uudistuksen vuoksi organsaatiot joutuivat tarkistamaan ja uudistamaan tietosuojakäytäntöjään. Tässä työssä tutustutaan sekä yleisesti tietosuojan historiaan, tietosuoja-asetuksen tuomiin muutoksiin verrattuna entisiin säädöksiin, että TKO-äly ry:n tietosuojan auditointiin ja uudistuksen myötä tulleisiin muutostarpeisiin. Uuden tietosuoja-asetuksen muutokset suhteessa entiseen tietosuojadirektiiviin ja Suomen tietosuojalakiin aiheuttivat muutoksia niin yhdistyksen tietojärjestelmiin, tietosuojaselosteisiin kuin toimintatapoihinkin. Auditoinnin tulosten perusteella tehdyissä muutostöissä yhdistyksen järjestelmien pääsyoikeuksia rajattiin, henkilötietojen jakelua ja julkaisukanavia muokattiin.

\end{abstract}

\mytableofcontents

\section{Johdanto}

Euroopan Unionin (EU) tietosuoja perustui direktiiviin vuodelta 1995, sekä sitä täydentäviin muihin sopimuksiin vuosilta 2000-2016. Suomessa voimassa ollut henkilötietolaki pohjasi näihin direktiiveihin. Toukokuussa 2018 voimaan astui uusi tietosuoja-asetus, joka toi mukanaan tarkennuksia ja laajennuksia voimassa oleviin säännöksiin, mutta myös uusia vastuita rekisterinpitäjille ja uusia oikeuksia rekisteröidyille.

Ennen uuden tietosuoja-asetuksen voimaan astumista tein Helsingin yliopiston tietojenkäsittelytieteen ainejärjestö TKO-äly ry:lle tietosuoja-auditoinnin. Auditoinnissa kävimme yhdessä asiakkaan kanssa läpi yhdistyksen tietojärjestelmät ja asiakirjat. Auditoinnin tulosten myötä tehtiin myös parannuksia asiakkaan kanssa, saattaen yhdistyksen toiminta uuden tietosuoja-asetuksen mukaiselle tasolle. Tämän työn tarkoituksena on esitellä asetuksen myötä tulleita muutoksia,niiden vaikutuksia edellä mainittuihin järjestön dokumentaatioon, järjestelmiin ja käytäntöihin, sekä miten muutokset tehtiin auditoinnin yhteydessä.

\section{EU:n tietosuojan historiaa}

Euroopan Unionin tietosuoja perustui aiemmin direktiiviin 95/46/EY, yksilöiden suojelusta henkilötietojen käsittelyssä ja näiden tietojen vapaasta liikkuvuudesta \cite{eu95}. Sen perusta oli kuitenkin laadittu paljon aikaisemmin \cite{tikkinen}. Vuonna 1970 säädettiin Saksan Hessenissä ensimmäinen tietosuojasäädös, ja vuonna 1973 Ruotsissa otettiin käyttöön vastaava asetus. Samoin vuonna 1973 Amerikan Yhdysvaltain hallitus sääti Fair Information Practices (FIPs) \cite{tikkinen}. 1981 Euroopan neuvosto laati sopimuksen henkilönsuojasta henkilökohtaisen datan käsittelyyn liittyen (sopimus 108), ja samoihin aikoihin OECD laati FIPsiin perustuvaa omaa sopimustaan. Näiden pohjalta moni eurooppalainen maa omaksui omat tietosuojalakinsa \cite{tikkinen}, ja näihin kaikkiin pohjautui EUn direktiivi 95/46/EY sekä nyt voimassa oleva asetus 2016/679 (yleinen tietosuoja-asetus). 

Edellinen voimassa ollut direktiivi on vuodelta 1995, mutta se ei ollut ainut voimassa oleva säädös. Sitä täydennettiin vuonna 2000 laadituilla Safe Harbor -periaatteilla Amerikan Yhdysvaltain ja Euroopan Unionin välillä. Safe Harbor -periaatteet julistettiin pätemättömäksi 2015, ja niiden tilalle laadittiin vuonna 2016 voimaan astunut Privacy Shield -sopimus \cite{privacy,tikkinen}. Näiden lisäksi vuonna 2002 direktiiviä 95/46/EY täydennettiin toisella, 2002/58/EY -direktiivillä, joka keskittyi henkilötietojen käsittelyyn ja yksityisyyden suojaan \cite{eu2002,tikkinen}. Näihin direktiiveihin täydennyksineen pohjautui myös Suomen henkilötietolaki \cite{henkilotieto}.

Yleinen tietosuoja-asetus astui voimaan toukokuussa 2018, mutta se on ollut tekeillä jo vuodesta 2009. Euroopan komissio julkaisi ensimmäisen virallisen ehdotuksen 2012, ja se lopulta hyväksyttiin 2016 \cite{eu2016,tikkinen}.


\newpage
\section{Aiemmat säädökset}

Koska käsittelemme eritasoisia Euroopan Unionin säädöksiä, on syytä avata säädösten eri tasoja ja merkityksiä.
\begin{itemize}
\item Asetus on sitova säädös, joka koskee kaikkia jäsenvaltioita. Asetus on siis kaikkein voimakkain keino. \cite{europa}
\item Direktiivillä määritellään tavoitteet, joihin kaikkien EU-maiden on yllettävä. Maat saavat itse päättää laeista, joilla direktiivin tavoitteet toteutetaan. \cite{europa}
\item Päätökset sitovat vain niitä, joille ne on osoitettu. Kaikki päätökset eivät esimerkiksi koske kaikkia jäsenmaita \cite{europa}
\end{itemize}
Tietosuoja perustui aiemmin direktiiviin ja sen täydennyksiin, joka antoi jäsenvaltioille vapauden määrittää tarkemmat yksityiskohdat paikallisella lainsäädännöllä. Uusi yleinen tietosuoja-asetus on kuitenkin sitovampi, jolloin tietosuojaa saatiin yhtenäistettyä \cite{eu2016,europa}. Aiemmin maiden välillä on ollut mahdollisuus olla suuriakin eroavaisuuksia tietosuojan vaatimusten välillä. Koska entinen tietosuoja ei koostunut vain yhdestä säädöksestä, säädöksiä käsitellään relevanssijärjestyksessä laatimisjärjestyksen sijaan.

Tietosuojan tason oletetaan usein olevan "riittävä" ja "asianmukainen" \cite{eu95,eu2016}. Nämä hyvin tulkinnanvaraiset termit ovat sellaisenaan eri säädöksistä, eikä niiden tarkempaa kuvausta anneta. Koska lakien soveltaminen perustuu aina laintulkintaan, on riittävä ja asianmukainen taso eri oikeusasteiden tehtävänä määrittää. 

\subsection{Edellinen tietosuojadirektiivi}

Euroopan parlamentin ja neuvoston direktiivi 95/46/EY, yksilöiden suojelusta henkilötietojen käsittelyssä ja näiden tietojen vapaasta liikkuvuudesta (tietosuojadirektiivi 1995), on nykyisen tietosuoja-asetuksen suora edeltäjä \cite{eu95,eu2016}. Nimensä mukaisesti tällä tarkoitettiin ensimmäistä kokonaisvaltaista säädöstä tietosuojasta. Tällöin ei kuitenkaan tiedetty, mihin kaikkeen sitä tultaisiin soveltamaan \cite{eu2016}. Tässäkin direktiivissä oli kuitenkin jo huomioitu yksilön tietosuoja, sekä niissä tapauksissa kun rekisterinpitäjä on EU:n sisällä, että sen ollessa EU:n ulkopuolella. Datan siirtäminen kolmansiin maihin, eli EU:n ulkopuolelle, oli sallittua jos kyseisessä kolmannessa maassa turvattiin tietosuojan riittävä taso \cite{eu95,safeharbor}.

Henkilökohtaisen datan käsittelystä oli selvä linja jo tässä direktiivissä. Tietoja henkilöstä saisi kerätä riittävästi ja olennaisissa määrin, eikä niitä saisi olla liikaa. Tämä on hyvin yhtenevä myös uuden tietosuoja-asetuksen kanssa \cite{eu95,eu2016,tikkinen}. On myös huomattavaa, että direktiivissä on otettu huomioon telepalvelu ja sähköposti, mutta internetiä ei, sillä sen käyttötarkoitus ja levinneisyys oli tuolloin huomattavasti suppeampi \cite{eu95}.

\newpage
\subsection{Täydentävä direktiivi}

Euroopan parlamentin ja neuvoston direktiivi 2002/58/EY, sähköisen viestinnän tietosuojadirektiivi (tietosuojadirektiivi 2012) luotiin täydentämään tietosuojaa, jonka pohja oli laadittu direktiivissä 95/46/EY. Tämä ei kuitenkaan korvannut aiempaa direktiiviä vaan nimenomaisesti täydensi sitä. Direktiivin tarkoitus oli "turva[ta] luonnollisten henkilöiden henkilötietojen käsittelyä koskevat oikeudet ja [...] oikeutensa yksityisyyden suojaan, jotta henkilötietojen vapaa liikkuvuus yhteisössä voidaan turvata." \cite{eu2002} 

Verrattuna aiempaan direktiivin, tässä jo kirjallisesti ja nimenomaisesti huomioitiin internet ja kuinka se mullistaisi perinteisiä markkinarakenteita, tuoden uusia mahdollisuuksia, mutta myös uusia riskejä \cite{eu2002}. Tässä direktiivissä luotiin myös selkeä pohja nykyiselle tietosuoja-asetukselle, painottaen käyttäjän suostumusta datan käsittelyn oikeuttavana asiana \cite{eu2002,tikkinen}. Direktiivissä otettiin huomioon myös internetin mukana tulleita erityispiirteitä, kuten evästeet ja niiden käyttötarkoitukset mahdollisessa käyttäjän datan tallentamisessa, sekä käyttäjän oikeus kieltäytyä "evästeen tai vastaavan menetelmän tallentaminen päätelaitteelleen" \cite{eu2002}. Tämä direktiivi jäi voimaan sellaisenaan uuden tietosuoja-asetuksen voimaantulon jälkeenkin \cite{eu2016}.

\subsection{Safe Harbor ja Privacy Shield}

Komission päätöksen 2000/520/EY, eli Safe Harbor -periaatteiden, tarkoituksena oli taata yksityisyyden suoja henkilötietoja siirrettäessä EU:sta Amerikan Yhdysvaltoihin \cite{safeharbor} ja erityisesti ottaa huomioon yksityisyyden suojan erot EU:n ja Yhdysvaltojen välillä. Sekä EU:ssa että Yhdysvalloissa oli paikallisia lainsäädäntöjä, EU:ssa eroja oli valtioidenkin välillä, ja Yhdysvalloissa osavaltioiden välillä \cite{tikkinen}. Yhdysvalloissa käytettiin myös alakohtaista lähestymistapaa, jossa yhdistyivät lainsäädäntö, sääntely ja itsesääntely \cite{safeharbor}. 

Periaatteet laadittiin Yhdysvaltojen kauppaministeriön ehdotuksen pohjalta \cite{safeharbor,tikkinen}ja EU:n komission päätös vahvisti periaatteet \cite{safeharbor}. Yhdysvaltain kauppaministeriö julkaisi useita asiakirjoja, joiden pohjalta nämä periaatteet sovittiin. Asiakirjoissa määritettiin muun muassa Yhdysvaltain lainsäädäännön mukaiset täsmälliset valtuudet sekä kriteerit, miten ja mihin periaatteita sovelletaan \cite{safeharbor}. Yhdysvaltojen lainsäädäntöä sovellettiin organisaatioihin, jotka ovat sitoutuneet safe harbor -periaatteisiin, poikkeuksena ne organisaatiot, jotka sitoutuivat yhteistyöhön Euroopan tietosuojaviranomaisten kanssa. Periaatteista ja säännöistä sovellettiin kaikkia, ellei toisin mainittu. \cite{safeharbor}.

Organisaatio, joka käsitteli henkilökohtaista dataa, takasi itse oman tietoturvansa riittävyyden, ja niitä valvoi yhdysvaltain viranomaiset \cite{safeharbor,tikkinen}. Organisaatiot saivat päättää täysin vapaaehtoisesti, liittyvätkö Safe Harbor -järjestelmään ja ne voitiin hyväkysä järjestelmään eri tavoin. Riittävät yksityisyyden suojan takaavat toimet pystyi takaamaan esimerkiksi laatimalla kirjallisen sopimuksen EU:sta tietoa siirtävien osapuolten kanssa \cite{safeharbor}.

Euroopan unionin tuomioistuin julisti lokakuussa 2015 antamassaan tuomiossa päätöksen 2000/520/EY pätemättömäksi \cite{privacy}. "[T]uomioistuin katsoi, että komissio ei ollut todennut [...] päätöksessä, että Yhdysvallat takaa tietosuojan riittävän tason sisäisen lainsäädäntönsä tai kansainvälisten sitoumustensa johdosta" \cite{privacy}. Yhdysvaltain tietosuojan tason ei katsottu täyttävän vaadittua perusvapauksien ja -oikeuksien tasoa, erityisesti sallien Yhdysvaltain viranomaisten puuttumisen yksityisyydensuojaan \cite{privacy,tikkinen}. Tämän tuomion myötä oli tarve uudelle sopimukselle EU:n ja Yhdysvaltain välille. Valmistelut aloitettiin jo itseasiassa 2014, ja 2016 uusi Komission täytäntöönpanopäätös 2016/1250, Privacy Shield -järjestely astui voimaan.

Kun Safe Harbor -periaatteet oli tehty Yhdysvaltain kauppaministeriön ehdotuksen pohjalta, uusi Privacy Shield -järjestely on tehty tiiviimmässä yhteistyössä EU:n komission ja Yhdysvaltain viranomaisten kanssa. EU:n komissio on analysoinut Yhdysvaltojen lainsäädäntöä tätä järjestelyä tehdessä \cite{privacy}. Verrattuna Safe Harbor -periaatteisiin, Privacy Shield -järjestelyssä on erikseen otettu huomioon Yhdysvaltain viranomaisten pääsy Yhdysvaltoihin siirrettyihin henkilötietoihin, johtuen pääasiassa edellisen sopimuksen mitätöineestä oikeudenkäynnistä \cite{privacy,tikkinen}. Yhdysvallat sitoutui luomaan uuden valvovan viranomaisen puhtaasti tätä tarkoitusta valvomaan \cite{safeharbor}.

Kuten edeltäjässään, myös tässä sopimuksessa dataa käsittelevät organisaatiot itsevarmentavat sitoutumuksensa Privacy Shieldin periaatteisiin, jotka on antanut Yhdysvaltain kauppaministeriö \cite{privacy}. Soveltamista on kuitenkin muutettu siten, että henkilötietoja käsittelevien yhdysvaltalaisten tahojen "[...] on oltava velvollisia toimimaan ainoastaan EU:n rekisterinpitäjän ohjeiden perusteella [...]" \cite{privacy}. Näin velvollisuus toimia sopimuksen mukaisesti ei ole ainoastaan Yhdysvaltain kauppaministeriön valvonnassa, vaan myös EU:ssa on viranomainen sopimusta varten.

Uusi tietosuoja-asetus ei ota kantaa Privacy Shield -järjestelyyn, sillä asetus on säädetty ennen kuin Privacy Shield astui voimaan. Uuden asetuksen mukaisesti kuitenkin kaikki viittaukset asiakirjoissa direktiiviin 95/46/EY pidetään viittauksina uuteen asetukseen \cite{eu2016}. Täten Privacy Shield ei myöskään vanhentunut uuden asetuksen astuessa voimaan, vaan sen katsotaan viittavan tietosuoja-asetukseen ja täydentävän sitä.


\newpage
\section{Yleinen tietosuoja-asetus}

Euroopan parlamentin ja neuvoston asetus 2016/679, yleinen tietosuoja-asetus, on nykyinen voimassa oleva tietosuoja-asetus, joka tuli voimaan sellaisenaan jokaisen Euroopan Unionin jäsenvaltion lainsäädäntöön. EU:n aiempi tietosuoja perustui direktiiviin 95/46/EY ja sen täydennyksiin. Direktiivi oli säädetty aikana, jolloin internet ei ollut vielä yleistynyt. Uuden asetuksen saatteessakin todetaan, että “Direktiivin 95/46/EY tavoitteet ja periaatteet ovat edelleen pätevät, mutta sen avulla ei ole pystytty estämään tietosuojan täytäntöönpanon hajanaisuutta eri puolilla unionia,” \cite{eu2016} joten uusi Euroopan parlamentin ja neuvoston asetus (EU) 2016/679, luonnollisten henkilöiden suojelusta henkilötietojen käsittelyssä sekä näiden tietojen vapaasta liikkuvuudesta ja direktiivin 95/46/EY kumoamisesta (yleinen tietosuoja-asetus)" annettiin huhtikuussa 2016 ja astui voimaan toukokuussa 2018 \cite{eu2016}. Uusi tietosuoja-asetus ei kumoa kaikkia voimassa olevia direktiivejä, sopimuksia ja sääntöjä, vain direktiivin 95/46/EY \cite{eu2016}.

Yleinen tietosuoja-asetus on julkaistu EU:n virallisessa lehdessä. Itse asetusta edelsi sovellusohjeet, joissa huomioitiin muut voimassa olleet säädökset ja mahdolliset poikkeukset tai ristiriidat. Tietosuoja-asetus on jaettu yhteentoista lukuun, joista jokainen käsittelee yhtä loogista kokonaisuutta \cite{eu2016}. Tässä työssä kuitenkin yhdistetään lukuja työn kannalta relevantteihin suurempiin kokonaisuuksiin, eikä jokaista lukua käsitellä yhtäläisellä tarkkuudella toisiinsa nähden.

\subsection{Yleiset säännökset ja periaatteet}

Asetuksen alussa määritellään yleiset säännökset eli käyttötarkoitus, soveltamisalat ja määritelmät käytetyille termeille. Tämän lisäksi määritellään henkilötietojen käsittelyä, lainmukaisuutta ja suostumusta koskevat periaatteet. Näiden lisäksi on erikseen periaatteet erityisryhmille, joita ei voi yleistää \cite{eu2016,tikkinen}. Näitä ovat esimerkiksi periaatteet lapsen suostumukselle ja vanhempien vastuunkanto sekä tietojen käsittelyn kielto tapauksissa, joissa ilmenee esimerkiksi käyttäjän etninen alkuperä, uskonnollinen tai filosofinen vakaumus, tai biometristen tietojen käsittely henkilön yksiselitteistä tunnistamista varten \cite{eu2016,tikkinen}.

\subsection{Rekisteröidyn oikeudet}

Rekisteröidyllä, eli henkilöllä jonka tietoja kerätään ja käsitellään, on nykyisen asetuksen myötä tarkemmin määritettyjä perusoikeuksia henkilötietoihin liittyen. Rekisterinpitäjän on informoitava rekisteröidyttäessä selkeästi, mihin käyttäjän dataa käytetään. Tämän lisäksi rekisteröityneen niin pyytäessä, tulee rekisterinpitäjän toimittaa rekisteröityneelle tiedot siitä, mitä datalle on tehty. Läpinäkyvyyttä painotetaan vahvasti. Esimerkiksi, rekisterinpitäjän yhteystiedot tulee olla julkiset, kerättyjen henkilötietojen käsittelyn tarkoitus ja oikeusperuste pitää olla käyttäjän saatavilla sekä tieto datan mahdollisesta siirtämisestä kolmansiin maihin ja tähän liittyvät suojatoimet \cite{tikkinen}.

Rekisteröidyllä on kerätyn datan saamisen lisäksi oikeus oikaista hänestä kerätyssä datassa olevat epätarkat ja virheelliset tiedot. Rekisterinpitäjän pitää toimia tällaisessa tapauksessa ilman aiheetonta viivytystä. Jos rekisteröity niin haluaa, voidaan tällaisella oikaisulla myös täydentää hänestä kerättyjä henkilötietoja. \cite{eu2016}. Tämän lisäksi käyttäjällä on oikeus tulla unohdetuksi, eli vaatia rekisterinpitäjää poistamaan häntä koskevat henkilötiedot ilman aiheetonta viivytystä. Unohtamiseen tarvitsee kuitenkin olla peruste, joita on muun muassa datan käyttötarkoituksen vanheneminen, tietojen käsittelyn lainvastaisuus tai jos rekisteröity peruuttaa suostumuksensa. Perusteeksi riittää mikä tahansa asetuksessa säädetty yksinään \cite{eu2016,tikkinen}.

\subsection{Rekisterinpitäjä ja henkilötietojen käsittelijä}

Rekisterinpitäjä on se taho, joka omistaa henkilötietorekisterin tiedot. Rekisterinpitäjä on aina oikeushenkilö, mutta ei välttämättä luonnollinen henkilö \cite{eu2016}. Rekisterinpitäjä voi toimia EU:n sisällä, mutta myös kolmansista maista. Jälkimmäisessä tapauksessa rekisterinpitäjän on nimettävä kirjallisesti edustaja EU:n aluetta varten. Rekisterinpitäjän on toteutettava toimintansa vastuullisuus ja tekniset toteutukset. Tätä varten rekisterinpitäjä voi itsesertifioitua, eli taata velvollisuuksien noudattaminen. Takaukseen kuuluu muun muassa tiedon turvaaminen sekä tiedon tarpeellisuuden varmistaminen. Rekisterinpitäjä ei saa kerätä ylimääräistä dataa käyttäjästä, eikä säilyttää sitä tarpeettoman pitkään \cite{eu2016,tikkinen}. 

Henkilötietojen käsittelijä voi olla luonnollinen tai oikeushenkilö, joka toimii rekisterinpitäjän nimissä. Käsittelijä ei saa käyttää toisen henkilötietojen käsittelijän palveluksia ilman rekisterinpitäjän lupaa. Henkilötietojen käsittelijän tulee noudattaa asetuksessa säädettyjä määräyksiä, kuten rekisterinpitäjän auttaminen datan käsittelyyn liittyvät velvollisuudet, käsittelyn päättymisen jälkeen palauttaa tai poistaa datan ja sallii toimintansa auditoinnit \cite{eu2016,tikkinen}.

\subsection{Henkilötietojen siirto kolmansiin maihin tai kansainvälisille järjestöille}

Henkilötietojen siirto EU:n ulkopuolelle on sallittua vain, jos rekisterinpitäjä ja henkilötietojen käsittelijä noudattavat kaikkia asetuksessa annettuja edellytyksiä. Näin varmistetaan, että luonnollisten henkilöiden tietosuojan taso ei vaarannu. \cite{eu2016}. Vaatimukset kattavat niin eettisiä periaatteita kuin laillisia vaatimuksia, mutta näiden lisäksi myös asetuksen myötä tulleita vaateita, kuten sertifiointia. Erikseen myös todetaan, että tiettyihin kolmansiin maihin, tietyille organisaatioille tai järjestöille ei henkilötietoja saa siirtää, sillä niiden tietosuojan taso ei ole riittävä. \cite{eu2016}.

\subsection{Riippumattomat valvontaviranomaiset, yhteistyö ja yhdenmukaisuus}

Asetuksessa määritellään viranomaiset, joiden tehtävänä on valvoa asetuksen toimeenpanoa. Jokainen jäsenvaltio velvoitetaan osoittamaan riippumaton viranomainen hoitamaan tietosuoja-asetuksen soveltamista ja valvontaa. Valvovan viranomaisen tulee olla aidosti riippumaton, niin taloudellisesti kuin toiminnallisesti \cite{eu2016}. Näiden viranomaisten toimivalta, tehtävät ja valtuudet on myös osoitettu asetuksessa. Koska kyseessä on EU:n laajuinen asetus, velvoitetaan nämä viranomaiset toimimaan yhteistyössä keskenään yhtenäisen linjan ja tulkinnan saavuttamiseksi \cite{eu2016}.

Myös aiemmin määritettyjen kansallisten viranomaisten keskinäiselle yhteistyölle on asetettu vaatimukset. Valvovilla viranomaisilla on velvollisuus keskinäiseen avunantoon ja mahdollisuus yhteisiin operaatioihin tai tietojen vaihtoon, sekä yhdenmukaisuuteen niin menettelyissä kuin tehtävissäkin. Näiden lisäksi määritellään riippumaton Euroopan tietosuojaneuvosto, joka toimii kansallisten valvontaviranomaisten asetuksen sovellusta yhdenmukaisesti EU:n alueella \cite{eu2016}.


\subsection{Oikeussuoja, erikoistilanteet, täytäntöönpano ja loppusäännökset}

Rekisteröidyllä on oikeus tehdä valitus valvontaviranomaiselle ja rekisterinpitäjää tai henkilötietojen käsittelijää vastaan. Rekisteröity voi tehdä näin joko itse tai edustettuna. Rekisterinpitäjällä ja henkilötietojen käsittelijällä on oikeus oikeuskeinoihin valvontaviranomaista vastaan. Asetuksessa säädetään myös seuraamukset ja sanktiot, mitä edellä mainituista tai muuten asetuksen rikkomiseen johtavista toimista seuraa \cite{eu2016}.

Asetuksessa ja sen saatteissa käsitellään poikkeuksia, kuten tieteellinen tutkimus, virallisten asiakirjojen julkisuus, työsuhteeseen liittyvä henkilötietojen käsittely tai uskonnolliset yhdistykset. Myös yleinen etu, tilastointi ja kansalliset erot kuten henkilötunnuksen muoto on huomioitu. Erikseen huomioidaan vielä jäsenvaltioiden sisäinen salassapitovelvollisuus omana poikkeuksenaan \cite{eu2016}.

Erikseen on määritetty myös asetuksen täytäntöönpano ja suhde muihin säädöksiin. Tällä vahvistetaan direktiivin 95/46/EY kumoaminen uudella tietosuoja-asetuksella, huomioiden muut voimassa olevat säädökset. Näin esimerkiksi Privacy Shield tai direktiivi 2002/58/EY jäävät voimaan, ja viitteet kumottavaan direktiiviin pidetään jatkossa viitteenä uuteen tietosuoja-asetukseen. Huomattava uudistus on säädetty artikla tietyin väliajoin suoritettavista arvioinneista asetusta kohtaan, jolla pyritään varmistamaan ajantasaisuus ja yhdenmukaisuus \cite{eu2016}.

\newpage
\section{Tärkeimmät muutokset}

Yleinen tietosuoja-asetus on luotu nimenomaisesti henkilötietojen suojeluun ja niiden vapaaseen liikkuvuuteen. Tässä työssä keskitytään erityisesti siihen, mitä muutoksia uusi asetus toi mukanaan ja millaisia toimenpiteitä sen myötä piti tehdä, jotta TKO-äly ry:n henkilörekisterit, tietojärjestelmät sekä henkilötietojen käsittelykäytänteet saataisiin ajantasalle. Käytännön vertailua eri säädösten välillä ei varsinaisesti tehty, sillä tarkastelun kohteena oli uusi asetus sekä tietojärjestelmien ja dokumentaation päivitys sen mukaiseksi.

Ensimmäinen huomattava muutos suhteessa entiseen on asetuksen vaikutusalue. Kun entinen direktiivi rajautui koskemaan toimijoita vain EU:n alueella, uusi asetus laajensi vaikutusaluetta koskemaan myös toimijoita EU:n ulkopuolella, jos ne käsittelevät EU:n kansalaisten henkilötietoja. Myös asetukseen liittyvät periaatteet ovat laajemmat kuin aiemmin. Uusi asetus toi mukanaan läpinäkyvyyden vaatimuksen, datan minimalisoinnin sekä lapsiin kohdistuvan henkilösuojan erityispiirteet \cite{eu2016,tikkinen}.

Läpinäkyvyyttä korostaa jo aiemmin mainitut vaatimukset rekisterinpitäjän tietojen julkisuudesta, mutta myös rekisteröidyn oikeus saada itselleen hänestä kerätty data ja niiden muokkaus-, päivitys- ja poisto-oikeus ovat laajemmat kuin aiemmin. Suomen henkilötietolaissa näitä vastaavat oikeudet rekisteröidyllä kuitenkin olivat jo olemassa ja selkeästi kirjattu lakiin \cite{henkilotieto}. Tämä on kuitenkin merkittävä uudistus Euroopan unionin tasolla, sillä tämä toi datan omistajuuden rekisteröidylle käyttäjälle, rekisterinpitäjän sijaan \cite{eu2016}. Kuten edeltäjissäänkin, myös uudessa asetuksessa käyttäjän suostumuksella on hyvin vahva merkitys siihen, miten hänestä kerättyä dataa voidaan käyttää, mutta nykyään hänellä on vielä suurempi vaikutus siihen, mitä dataa hänestä voidaan kerätä \cite{eu2016,rikke}.

Datan suojaus on myös tarkemmin määritelty kuin nykyisin. Entistä direktiiviä kritisoitiin sen riittämättömyydestä ja hajanaisuudesta. Uuden asetuksen myötä suojaus pyritään saamaan koko unionin alueella yhtenevälle, riittävälle tasolle. Vaikka asetus ei suoranaisesti kerro, miten esimerkiksi tekninen suojaus tulee toteuttaa, velvoite tehdä niin kuitenkin on \cite{eu2016,barb,tikkinen}.

Erityisen merkittävä parannus on henkilötietojen siirron suojaus, kun puhutaan kolmansista maista ja monikansallisista organisaatioista. Uudessa asetuksessa on laaja lista vaatimuksia, jotka henkilötietoja käsittelevän organisaation tulee täyttää, ennen kuin sillä on oikeus siirtää dataa EU:n ulkopuolelle. Tärkeimpiä vaatimuksia ovat siirron tarpeellisuus ja välttämättömyys, siirtokohteen tietosuojan riittävä taso ja siirron toteuttaminen asianmukaisia suojatoimia soveltaen \cite{eu2016}. 

Datan kerääminen oli ennen vapaampaa kuin nykyisen tietosuoja-asetuksen ollessa voimassa. Tietosuoja-asetuksen myötä henkilötietoja keräävä yritys joutuu tarkasti määrittämään, mitä dataa ja mihin tarkoitukseen kyseistä dataa kerätään, käytetään, ja kuinka kauan sitä säilytetään. Jos näitä ehtoja ei täytetä, sanktiot ovat huomattavat \cite{eu2016}. Suomen henkilötietolaissa oli säädetty uhkasakko henkilötietojen käsittelyn rikkomisesta, mutta sen suuruus oli harkinnanvarainen \cite{henkilotieto}.

\subsection{Muutosten vaikutus TKO-äly ry:hyn}

Koska auditoinnin kohteena oleva järjestö toimii Suomessa ja jäsenistön suurin osa on täysi-ikäisiä EU:n kansalaisia, vaikutusalueen laajuuden muutos ei varsinaisesti koske auditointia, mutta on kuitenkin hyvä tiedostaa, sillä esimerkiksi pilvipalveiden käyttämiseen tiedon edes väliaikaiseen tallentamiseen tämä vaikuttaa, joten tähän reagointiin myös tämän auditoinnin yhteydessä. Kaikki rekisteröidyt eivät ole välttämättä EU:n kansalaisia, mutta yhdenvertaisuuden ja järjestelmien yksinkertaistamisen vuoksi kaikkia käsitellään yhdenvertaisena.

TKO-äly ry:n tapauksessa läpinäkyvyyden lisäämiseksi muun muassa henkilörekistereiden rekisteriselosteet laadittiin uudelleen vastaamaan uuden tietosuoja-asetuksen vaatimuksia. Lisäksi tapahtuma-ilmoittautumisten yhteydessä rekisteröity voi nykyisin päättää, saako hänen nimensä julkaista tapahtumaan ilmoittautuneiden listalla, kun aiemmin nimet julkaistiin aina.

TKO-äly ry:n tietosuojalausekkeissa on nykyisin maininta kolmansista osapuolista, sillä tietosuoja-asetus vaikuttaa esimerkiksi asiakirjojen säilyttämiseen kolmansien osapuolien pilvipalveluissa, kuten Google Drivessä. Järjestö myös toimii HYYn alaisuudessa, jolloin järjestöllä on tarvittaessa velvollisuus kertoa jäsentensä opiskelijastatus ylioppilaskunnalle. Koska tämäkin katsotaan kolmannelle osapuolelle luovuttamiseksi, on se huomioitu uudistetussa dokumentaatiossa.

Sakko koskee kaikkia rekisterinpitäjiä, joten se toimi hyvin pelotteena tietosuoja-asetuksen noudattamiseen ja oli myös osasyy tämän auditoinnin tekemiselle. Vaikka onkin epätodennäköistä, että pienelle opiskelijajärjestölle langetettaisiin sakkoa tietosuojarikkomuksen perusteella, TKO-äly ry:n edustajille ja jäsenistölle on tietosuojan noudattaminen tärkeää, joten asetuksen noudattaminen tarkalleen on luonnollista.

\newpage
\section{Auditointi käsitteenä}

Alunperin auditoinnilla tarkoitettiin tilintarkastusta vastaavaa toimintaa, joka on myöhemmin laajentunut muihin liiketoiminnan osa-alueisiin \cite{aditya,felley}. IT-auditoinnilla tarkoitetaan yleisesti tietojärjestelmien infrastruktuurin ja turvallisuuden tarkastelun lisäksi järjestelmien hallinnan ja päivittäisoperoinnin systemaattista arviointia \cite{felley}. Täysimittainen IT-auditointi on kuitenkin tarkoitettu nimenomaisesti yrityksen liiketoiminnan tarkoitusta varten \cite{aditya}. Vaikka auditointia varten on kehitetty erilaisia standardeja, joista yleisin on ISO 27001, järjestelmien monimuotoisuuden takia ei kaikkia auditointeja voida suorittaa samalla tavalla, vaan standardeissakin on muokkaamisen varaa \cite{dimond,felley}. Tällä kertaa tehty tietosuoja-auditointi poikkeaa siis hyvinkin paljon kokonaisvaltaisesta IT-auditoinnista, keskittyen vain yhteen osa-alueeseen ja sen vaikutuksiin. 

Auditoinnin rakenteesta tai suoritustavasta ei ole yhtä yhtenevää mielipidettä, eikä kaikille sovelluksille sopivaa yhtenevää standardia \cite{dimond,aditya,frost,felley}. Auditoinnin rakenne voi karkeimmillaan olla kolmiosainen: arviointi, tarkastelu ja raportointi, joista jokainen voi sisältää vaihtelevan määrän pienempiä osa-alueita \cite{dimond,felley}. Toisenlainen esitys on seitsemänosainen: suunnittelu, datan kerääminen, analysointi, arviointi, suositusten kommunikointi, niiden toteutus sekä jatkuva tiedon hallinta \cite{frost}. Toisaalta taas auditointia varten on myös kehitetty erilaisia visuaalisia runkoja ja protokollia, kuten seuraava kuva \ref{auditointi} osoittaa. Tätä nimenomaista protokollaa mukaellen suoritettiin myös tämän työn tietosuoja-auditointi.
\newline
\begin{figure}[H]
\includegraphics[width=0.5\textwidth]{Overview_of_audit_protocol.JPG}
\centering
\caption{Auditointiprotokolla \cite{dimond} }
\label{auditointi}
\end{figure}

Kuva \ref{auditointi} on jaettu kolmeen osaan, A, B ja C. Tämän protokollan mukaan A ei kuulu auditoinnin piiriin, B on osa auditoinnin strategista kehittämistä, ja C on itse auditoinnin suorittava vaihe \cite{dimond}. Nyt tehty tietosuoja-auditointi vastasi vahvasti kuvan esittämää strategiaa, kuitenkin soveltaen ja hieman kevyemmällä otteella. 

\section{TKO-äly ry:n tietosuoja-auditointi}

Ennen varsinaista auditointia arvioitiin ympäristö, eli mitä auditoidaan ja millä tavoin, kuten kuvassa \ref{auditointi}, \textit{"A1: assessing the environment"}. Kuvan kohdat \textit{"B1: Consideration of philosophy"} sekä \textit{"B2: Definition of objectives"} tehtiin käytännössä yhtenä vaiheena ja yhdessä asiakkaan kanssa, tarkastellen auditoinnin rajoitteita, tavoitteita ja vaatimuksia ennen auditoinnin suorittamista. Kohdat \textit{"B3: Assemble methods \& Define analysis"}, \textit{"B4: Consider presentation"} sekä \textit{"B5: Trial \& pilot"} jäivät auditoinnin suhteen vähemmälle huomiolle. Asiakas oli auditoinnissa hyvin vahvasti mukana ja käytännössä toteutti tekniset muutokset järjestelmiin, näiden osa-alueiden toteutus jäi enemmän asiakkaan kuin auditoijan vastuulle. Näitä kuitenkin käsitellään tässä työssä. Kohdat \textit{"B6: Assess \& redefine"} sekä \textit{"B7: Implement changes"} jakaantuivat sekä asiakkaan että auditoijan välille. Vaikka varsinaisista muutostöistä vastasikin pääasiassa asiakas, niiden asetusten mukaista toimintaa ja toteutusta tarkasteltiin erikseen auditoinnin yhteydessä, sekä sen jälkeen. 

Itse auditointityöhön viittavat kohdat, \textit{"C1: Implement methods"}. \textit{"C2: Apply analysis"} sekä \textit{"C3: Present results"} tehtiin hyvin yhtenäisenä ja manuaalisena toimenpiteenä. Auditointimetodina oli yhdistyksen järjestelmien ja dokumentaation manuaalinen läpikäynti ja tarkastelu tietosuojan näkökulmasta. Tämä tarkoitti käytännössä asiakkaan kanssa yhdessä tietojärjestelmien ja internetsivujen selaamista vaihe vaiheelta, sekä senhetkisten henkilörekisteriselosteiden lukemista. Tämän kartoituksen avulla saimme selville kokonaisvaltaisen kuvan muutosten määrästä, laadusta ja tarpeesta. Varsinaista tulosten esittelyä ei asiakkaalle tai asiakkaan käyttäjille tehty, vaan ne ilmenivät suoraan muuttuvina järjestelyinä yhdistyksen palveluissa.

\subsection{Auditoinnin vaiheet ja resurssit}

Vaikka työ tehtiinkin edellä esitetyn protokollan järjestyksessä, itse auditointia tehdessä ei noudatettu mitään varsinaista kaavaa, vaan työtä tehtiin siinä järjestyksessä kuin tarkastettavia järjestelmiä tuli mieleen. Kun auditoinnin tulisi olla looginen ja järkevä tarkastelu, nyt suoritettu oli enemmänkin tarpeen myötä tehty tarkastelu kuin auditointi määritelmällisesti.  Auditointi tehtiin hyvin manuaalisesti asiakkaan kanssa yhteistyössä,  ja tärkein työkalu olikin kommunikaatio asiakkaan edustajien kanssa. Asiakkaan kanssa käytiin läpi heidän käyttämänsä järjestelmät ja dokumentaatio. Jokaisen järjestelmän ja dokumentin läpikäynti oli aikaavievää mutta tarpeellista, kuten tulemme myöhemmin huomaamaan. TKO-äly ry:llä on vahva avoimen datan periaate, joka pyrittiin ottamaan huomioon auditoinnissa ja erityisesti tehtävissä muutoksissa mahdollisuuksien mukaan.

Työn valmistelu ja suorittaminen vaati erinäisiä resursseja, sisältäen niin työvoimaa kuin dokumentaatiopohjiakin. Auditointi tehtiin yhteistyössä asiakkaan edustajan, web-vastaava Aki Utoslahden kanssa. Hänen kanssaan käytiin järjestelmien eri osa-alueet lävitse vaiheittain, sekä tarpeen mukaan myös itse järjestelmien koodia. Yhdistyksen uudet tietosuojaselosteet on myös laadittu hänen kanssaan yhteistyössä. 

Tietosuojaselosteet laadittiin valmiiden esimerkkien pohjalle (Liitteet 5 ja 6). Dokumentaatiopohjat saatiin Tekniikan Akateemiset (TEK) -ammattiliiton sekä Helsingin yliopiston ylioppilaskunnan (HYY) yhteistyössä pitämän luennon myötä. Luennon tarkoitus oli tukea HYYn alaisuudessa toimivien järjestöjen tietosuoja-asetus-muutoksia, eli täten dokumentaatio oli vapaasti käytettävissämme TKO-älyn tietosuojalausekkeiden pohjaksi.

Auditointia tekemässä ei ollut lainoppineita, vaikka työssä tarvittiinkin ymmärrystä tietosuoja-asetuksesta sekä sen aiheuttamista seurauksista. Tämän vuoksi kysyin tarvittaessa apua Pykälä ry:ltä, joka tarjoaa oikeusapua Helsingin yliopiston järjestöille. Näiden kysymysten perimmäinen tarkoitus oli varmistaa oma näkemykseni tiettyjen seikkojen lainmukaisuudelle, ja yleensä Pykälä vahvistikin oman näkemykseni.

Varsinaisen muutostyön jälkeen vielä varmistettiin, että työ oli suoritettu oikein ja lopputulos on halutunlainen. Tämä tarkoitti sovellusten uudelleenläpikäyntiä sekä heti auditoinnin jälkeen, että myös myöhemmin tarkastelua asetuksen astuttua voimaan.


\subsection{Tilanne ennen auditointia ja sen jälkeen}

Auditoinnin ensimmäinen askel on luonnollisesti senhetkisen tilanteen kartoittaminen. TKO-älyn tapauksessa tämä tarkoitti järjestelmien, dokumenttien ja toimintatapojen systemaattista läpikäyntiä. Dokumentaation osalta tarkasteltiin henkilörekistereiden rekisteriselosteita. Järjestelmistä katsastettiin pääsyoikeudet eri palveluihin ja niiden rajaukset, esimerkiksi onko palvelu julkinen vai rajattu jäsenistölle tai vain hallitukselle. Toimintatavoista tarkasteltiin pääasiassa tiedottamista, julkaisua ja tiedon levittämistä, kuten pilvipalveluiden käyttöä tiedon tallentamisessa ja pöytäkirjojen julkaisutapoja.

Auditoinnin perimmäisenä tarkoituksena oli tarkastella edellämainittuja, mutta myös tuoda asiakirjat ja sovellukset ajantasalle vastaamaan nykyistä tietosuoja-asetusta. Seuraavassa perehdyn lähtötilanteeseen sekä auditoinnin myötä tehtyihin muutoksiin, keskittyen yhteen osa-alueeseen kerrallaan. 

\subsubsection{Dokumentaatio}

TKO-äly:n entinen rekisteriseloste oli kirjoitettu vanhan henkilötietolain mukaiseksi, mutta auditoinnissa tui ilmi, että siihenkin nähden se oli osittain puutteellinen (Liite 1) \cite{henkilotieto}. Silloinen seloste kattoi vain jäsenrekisterin, mutta tapahtumiin oli jo tuolloin mahdollista ilmoittautua ilman, että oli yhdistyksen jäsen. Täten rekisteriseloste ei kattanut koko toimintaa, vaikka oli tarkoitus. 

Auditoinnissa esiin tullut puutos rekisteriselosteen kattavuudessa korjattiin ensisijaisesti tekemällä useampi tietosuojaseloste (Liitteet 2 ja 3). Ensisijainen prioriteetti oli jäsenrekisterin tietosuojaseloste, jonka jälkeen perehdyimme tapahtumarekisterin selosteeseen. Jäsenrekisterin tietosuojaselosteen pohjana käytettiin HYY:n ja TEK:n mallia, ja tätä selostetta taas sellaisenaan pohjana myöhemmin tehtäville yhdistyksen tietosuojaselosteille. 

Uudet tietosuojaselosteet tehtiin lähestulkoon puhtaasti saatujen mallien ja esimerkkien mukaan, kuitenkin asiakkaan toiveet huomioon ottaen. Tietosuojalausekkeita laadittaessa ei ollut vielä tarkalleen tiedossa, kuinka tarkasti tietojen luovutusta tai säilyttämistä tulisi kuvata, joten myöskään vaihtoehtoja poiketa asiantuntijoiden ohjeista ei varsinaisesti harkittu. Uusissa tietosuojalausekkeissa on toisaalta kuvailtu asioita turhankin tarkasti. Esimerkiksi tietojen säilyttämisestä kirjanpidon tarkoituksiin ei tarvitsisi erikseen mainita tietosuojalausekkeessa, sillä muita lakeja koskevat poikkeukset on otettu huomioon tietosuoja-asetuksessa \cite{eu2016}. Asiakkaan toivomuksesta ja lausekkeiden laatimishetken ajankohdasta riippuen päätettiin olla ennemmin liian informativiisia, kuin ottaa riski sanktioista.

Vaikka selosteet sellaisenaan olivat olemassa jo ennen tietosuoja-asetuksen voimaantuloa, on asetuksen hengen mukaista, että niitä tarkastellaan aika-ajoin ja tarpeen mukaan päivitetään. Tämän vuoksi jäsenrekisterin selostetta on päivitetty joulukuussa 2018 kun huomattiin, että rekisteriin kohdistuu säännönmukaista tietojen luovutusta ylioppilaskunnalle. Tapahtumarekisterin selosteeseen ei ole tehty sittemmin muutoksia.

Tietosuojalausekkeiden lisäksi uudistettiin yhdistyksen tietosuojapolitiikka. Tämän dokumentin tarkoituksena on ohjeistaa henkilötietoja käsitteleviä henkilöitä oikeaoppiseen toimintaan sekä samalla avata yhdistyksen jäsenistölle yhdistyksen toimintatapoja henkilötietojen käsittelyssä. Kuten tietosuojaselosteita, tätäkin dokumenttia on päivitetty tarpeen mukaan julkaisun jälkeen.

Tietosuoja-asetuksen yksi päätavoite oli avoimuus tietojen käsittelystä. Koska asiakasyhdistyksen omiinkin toimintatapoihin kuului jo ennestään tiedon avoimuus, oli tämä vaatimus hyvin helposti toteutettavissa ja kaikki tietosuojaa koskeva dokumentaatio julkaistiin yhdistyksen internetsivuilla avoimesti saatavilla kaikille.

\subsubsection{Toimintatavat}

Ennen auditointia ja uutta tietosuoja-asetusta henkilötietojen käsittely oli TKO-äly:ssä melko vapaata. Periaatteessa tietoja pyrittiin käsittelemään silloisen lain mukaan. Kuitenkin yleinen tahto avoimesta tiedon jakamisesta, käytäntöjen osaamattomuus ja osittainen huolimattomuuskin olivat syinä siihen, ettei käsittely ollut aina tarpeeksi huolellista.

Esimerkkinä huolimattomasta käsittelystä oli kokouspöytäkirjojen julkaisu sellaisenaan yhdistyksen internetsivuilla. Viralliset kokouspöytäkirjat sisälsivät kokouksen osallistujaluettelon, jonka julkaisuun olisi jo entisen tietosuojalain aikaan tarvittu julkaisulupa osallistujilta. Myös henkilötietojen käsittely kokouksissa, kuten jäsenten erottamista tai jäseneksi hyväksymistä koskevat tiedot julkaistiin. Uudistusten myötä näistä tavoista luovuttiin, ja internetsivuilla julkaistaan karsittu versio kokouspöytäkirjasta, jossa ei lue henkilötietoja, vaan viralliset versiot arkistoidaan tietosuojaselosteen ja yhdistyslain mukaisesti.

Ennen auditointia henkilötietoja myös siirrettiin mahdollisesti EU/ETA-alueen ulkopuolelle, jäsenistölle kertomatta. Tämä tapahtui tallentamalla tietoja Google Drive -pilvipalveluun. Yhdistyksen yhteisestä pilvitallennustilasta ei kuitenkaan luovuttu uudistusten myötä, mutta tällaisesta tallenuksesta kerrotaan tietosuojalausekkeessa, kuten nykyisin on vaadittua.

Myös tapahtumailmoittautumisia muokattiin. Kun aiemmin kaikkien osallistujien nimet julkaistiin tapahtuman tietojen yhteydessä, uudistusten myötä osallistujan nimi julkaistaan vain, jos osallistuja itse antaa siihen luvan. Tietosuoja-asetuksen mukaisesti julkaisu tapahtuu nimenomaisesti henkilön omatoimisen valinnan mukaan, eli julkaisu ei ole automaattista.

\subsubsection{Tietojärjestelmät}

Yhdistyksen tietojärjestelmissä oli monenlaisia tietosuojapoikkeamia. Osa oli edellisen tietosuojalain hengen mukaisia, mutta muutettava uuden asetuksen mukaan. Toisaalta osa oli lainvastaisia jo ennestään, mutta esimerkiksi huomaamatta jäänyt avoimesti saataville.

Tietojärjestelmät olivat kuitenkin pääasiallisesti hyvässä kunnossa tietosuojan näkökulmasta. Muutokset joita vaadittiin olivat pieniä ja nopeita tehdä. Suuri osa muutoksista oli dokumentaation päivittämistä sivuille, mutta myös pääsyn rajausta tietoihin riippuen käyttäjän kirjautumisesta tai statuksesta. Muutoksia tehtiin myös hyvin iteratiivisesti. Esimerkiksi jo mainittu tapahtumatietojen julkaisu piilotettiin ensin kaikilta, jonka jälkeen annettiin hallituksen jäsenille pääsy tietoihin. Vasta viimeisenä versiona annettiin käyttäjille mahdollisuus julkaista oma nimensä niin halutessaan. Näin varmistettiin ensin lainmukaisuus, ja myöhemmin parannettiin toiminnallisuutta.

Auditoinnin myötä ilmeni erilaisia tulkintoja uuden tietosuoja-asetuksen soveltamisaloista. Tiukimman tulkinnan mukaan asiakkaan käyttämä tenttiarkisto, Tärpistö, muodosti henkilörekisterin sillä tenttitiedostojen nimet sisälsivät tentaattorien nimet. Tämän vahingossa luodun henkilörekisterin muuttaminen vaati jokaisen tiedoston uudelleen nimeämistä. Asiakkaan edustajan kanssa päädyimme ratkaisuun, jossa tentaattorien nimet tiedostonimissä korvataan tentin nimellä. Koska tiedostot olivat jo tenttien nimien mukaisessa kansiorakenteessa, oli itse tenttitiedostot nimettävä uudelleen polkunsa mukaisesti.

Uudelleennimeämiskomento on kirjoitettu BASH-skriptauskielellä. Lähtöaineistoa oli useampi sata tiedostoa, joten virheiden välttämiseksi sekä hakemistojen että tiedostojen nimistä poistettiin sekä ääkköset että välilyönnit, korvaten ne aakkosilla ja alaviivoilla.

Komento suoritettiin Tärpistön juuripolussa. Komennon alussa tulostettiin rivi jossa todetaan suorituspolku. Tämän tulosteen tarkoituksena oli varmistaa, että komento suoritetaan oikeassa paikassa. Vaikka komento ei poista tai korvaa alkuperäisiä kansioita, aiheuttaisi komennon ajaminen esimerkiksi järjestelmän juuressa suuren määrän uudelleennimettyjä tiedostoja, jotka veisivät levytilaa. Komennossa käydään polun sisältö läpi alihakemisto kerrallaan, ja alihakemistot tiedosto kerrallaan.

\begin{lstlisting}[language=BASH, inputencoding=utf8, literate= {ä}{{\"a}}1 {ö}{{\"o}}1  {Ä}{{\"A}}1 {Ö}{{\"O}}1 ]
#!/bin/bash

echo "We are here $PWD"

FILES=$PWD/*

for file in $FILES/*
do
  fn="${file##*/}"
  filepath=$(dirname "$file" | sed -e 's/ /_/g; s/-/_/g; s/ä/a/g; s/Ä/A/g; s/Ö/O/g; s/ö/o/g;' | cut -d '/' -f 5)
  mkdir -p $filepath
  mv "$file" $(dirname "$file" | sed -e 's/ /_/g; s/-/_/g; s/ä/a/g; s/Ä/A/g; s/Ö/O/g; s/ö/o/g;' )/${fn:0:7}$filepath${fn: -7}
done
\end{lstlisting}

Komennon rivillä 3 kerrotaan käyttäjälle, missä polussa komento suoritetaan. Tämä toteutettiin siksi, että käyttäjälle näkyisi varmistuksena, että hän on oikeassa paikassa suorittamassa komentoa. Käyttäjän palautetta tähän ei kuitenkaan pyydetty, sillä komento haluttiin pitää mahdollisimman yksinkertaisena. Käyttäjälle ei komennon suorituksen aikana näy muuta tulostetta ellei tule järjestelmävirheitä, joten tämän tulosteen lukemiseen on myös hyvin aikaa.

Seuraavaksi riveillä 7-11 käydään nykyisen sijainnin alikansiot läpi yksitellen ja luodaan uusi kansio, jonka nimessä on korvattu ääkköset aakkosilla ja välilyönnit alaviivoilla. Tämän jälkeen käydään läpi alikansion jokainen tiedosto uudelleennimeämistä varten, rivillä 12. Tiedostoja ei kuitenkaan suoranaisesti uudelleennimetty, vaan niistä luotiin aiemmin luotuun kansioon kopio uudella nimellä. 

Tiedostojen uusi nimi muodostettiin ottamalla alkuperäisestä tiedostonimestä ensimmäiset seitsemän merkkiä, uuden kansion nimi ja viimeiset seitsemän merkkiä. Tähän formaattiin päädyttiin tiedostojen alkuperäisen nimeämispolitiikan perusteella, jossa tiedoston alussa on kuusimerkkinen päivämäärä ja välilyönti, josta jälkimmäinen on nyt korvattu alaviivalla. Tiedostonimen keskimmäinen osa korvattiin kansionimellä, sillä juuri keskimmäinen osa oli tietosuojan kannalta ongelmallinen tentaattorin nimi. Nimen lopusta otetut seitsemän merkkiä olivat myös alkuperäisen nimeämispolitiikan mukaisia, kun lopussa oli tenttikoodi sekä tiedostopääte.

\newpage

Alkuperäisestä Tärpistöstä otettu katkelma:

\begin{lstlisting}[language=BASH,style=tree]
[...]
├── Tietorakenteet
│   ├── 121127_Floreen_EK.pdf
│   ├── 131209_Floreen_KK.pdf
│   ├── 131209_Huttunen_KK.pdf
│   ├── 140128_Floreen_EK.pdf
│   ├── 140224_Floreen_KK1.pdf
│   ├── 140408_Floreen_EK.pdf
│   └── 140408_Floreen_EK_en.pdf
├── Tietorakenteet\ ja\ algoritmit
│   ├── 141024_Laaksonen_KK1.pdf
│   ├── 150302_Floreen_VK1.pdf
│   ├── 150414_Tira_EK.pdf
│   ├── 150506_Floreen_KK2.pdf
│   ├── 150506_Floren_KK2.pdf
[...]
\end{lstlisting}

Vastaava osuus komennon suorituksen myötä oli seuraavanlainen:

\begin{lstlisting}[language=BASH,style=tree]
[...]
├── Tietorakenteet
│   ├── 121127_Tietorakenteet_EK.pdf
│   ├── 131209_Tietorakenteet_KK.pdf
│   ├── 140128_Tietorakenteet_EK.pdf
│   ├── 140224_TietorakenteetKK1.pdf
│   ├── 140408_Tietorakenteet_EK.pdf
│   └── 140408_Tietorakenteet_en.pdf
├── Tietorakenteet_ja_algoritmit
│   ├── 141024_Tietorakenteet_ja_algoritmitKK1.pdf
│   ├── 150302_Tietorakenteet_ja_algoritmitVK1.pdf
│   ├── 150414_Tietorakenteet_ja_algoritmit_EK.pdf
│   ├── 150506_Tietorakenteet_ja_algoritmitENG.pdf
│   ├── 150506_Tietorakenteet_ja_algoritmitKK2.pdf
[...]
\end{lstlisting}

Kuten huomataan, tieodostonimen viimeinen substituutio ei toiminut toivotulla tavalla. Koska kaikki alkuperäiset tiedostonimet eivät noudattaneetkaan samaa kaavaa, jäi muutamasta tiedostonimestä viimeinen alaviiva pois. Näitä tapauksia oli kuitenkin vain muutamia, joten tyydyimme tekemään korjaukset käsin, sen sijaan että itse komentoa olisi muokattu, vaikka muokkaus olisikin vaatinut vain viimeisen osan muokkaamista.

Komennon suorituksen jälkeen piti manuaalisesti poistaa alkuperäiset tiedostot. Tämä askel haluttiin jättää automatisoimatta varmuuden vuoksi, jotta voitiin tarvittaessa tarkastella alkuperäisiä tiedostoja uudelleennimettyihin, sekä jättää alkuperäiset tiedostot varmuuskopioksi siksi aikaa, kun tiedostot oli todistetusti siirretty takaisin palvelimelle.

Tietojärjelmät vaativat lopulta oletettua vähemmän muutostöitä. Vaikka muutoksia tehtiin useaan kohteeseen, olivat yksittäiset muutokset kuitenkin pieniä. Vaikka osa näistä olisi ollut tarpeellisia jo aiemmin, on huomattava ettei vastaavaa auditointia järjestelmiin ole aiemmin tehty, jolloin virheet jäävät helposti huomaamatta.

\newpage

\section{Yhteenveto}
Vaikka Suomessa oli jo ennen tietosuoja-asetusta hyvin säädelty tietosuojalaki, siinä ei ollut merkittävää sakon uhkaa laiminlyönnistä. Uuden asetuksen myötä tullut huomattava sanktio on saanut rekisterinpitäjät huomioimaan tietosuojan paremmin. Tämä näkyi myös tämän työn auditoinnin kohteessa. Järjestelmät olivat toteutettu pääosin vanhan lain hengen mukaisesti, mutta myös siihen nähden oli puutteita havaittavaissa. Auditoinnin myötä tällaiset epäkohdat havaittiin selkeästi ja niihin puututtiin samalla, kun järjestelmistä tehtiin uuden tietosuoja-asetuksen mukaisia. 

Uuden tietosuoja-asetuksen voimaantulon takia tehty auditointi ei ainoastaan tuonut esiin ongelmia, vaan samalla havaittiin perustason olleen kohtalaisen hyvällä tasolla. Auditoinnin tulosten myötä tehdyillä muutoksilla ei ainoastaan saavutettu uuden tietosuoja-asetuksen mukaisuutta, vaan samalla selkiytettiin toimintatapoja ja parannettiin jäsenistön tietosuojaa.

Tietosuoja-asetuksen voimaantuloa edeltävä auditointi oli asiakkaan tapauksessa tarpeellinen, mutta ei kriittinen. Suurin osa muutoksista oli pieniä ja nopeita toteuttaa. Tietosuoja-asetuksen voimaantulo aiheutti kuitenkin sen, että asiaan suhtauduttiin tarvittavalla vakavuudella, ja myös aikaisemmin muutoksia tarvinneet ominaisuudet saatiin nyt tuotua ajantasalle tietosuojan suhteen.


% --- References ---
\newpage
\nocite{*}
% one of these or  ...
%\bibliographystyle{plain}
% \bibliographystyle{acmfin}
\bibliographystyle{acm}
%\bibliographystyle{ieeetr}

% ... or this 
%\bibliographystyle{apalike}

\bibliography{my_references}

\lastpage



\pagestyle{empty}

\appendices

\internalappendix{1}{TKO-Äly ry:n aiempi rekisteriseloste}

\subsection*{1. Rekisterin pitäjä}
TKO-äly ry
PL 68
00014 Helsingin yliopisto

\subsection*{2. Rekisterin nimi}
TKO-äly ry:n jäsenrekisteri

\subsection*{3. Yhteyshenkilö}
Yhdistyksen jäsenvastaava
jasenvastaava@tko-aly.fi

\subsection*{4. Rekisterin käyttötarkoitus}
Jäsenrekisterin tietoja voidaan käyttää yhteydenpitoon jäsenten suuntaan, jäsenmaksujen tarkkailuun ja jäsenyyden tarkastamiseen.

\subsection*{5. Rekisterin tietosisältö}
Täydellinen nimi
Kotikunta
Sähköpostiosoite
Puhelinnumero
Jäsenmaksuun ja jäsenyyteen liittyvät tiedot

\subsection*{6. Rekisterin tietolähteet}
Jäsenhakemuksissa kerätyt tiedot sekä Helsingin yliopiston tietojärjestelmästä saadut tiedot.

\subsection*{7. Tietojen luovutus}
Tietoja voidaan käyttää yhdistyksen toiminnassa yhdistyksen hallituksen ja yhdistyksen sääntöjen määräämällä tavalla.

Yhdistyslain 11 § 2 momentin mukaisesti kaikilla yhdistyksen jäsenillä on oikeus tutustua tietoihin kaikkien jäsenten nimestä, kotipaikasta ja jäsenyydestä.

Tietoja ei luovuteta yhdistyksen ulkopuolelle.

\subsection*{8. Rekisterin suojauksen periaatteet}
Jäsenrekisteri sijaitsee yhdistyksen palvelimella tietokannassa. Palvelin on suojattu ulkopuoliselta käytöltä ja jäsentietojen käyttöä valvotaan. Pääsy rekisteriin on yhdistyksen ATK-järjestelmien ylläpitäjillä sekä jäsenvirkailijoilla. Rekisterinkäyttäjillä on henkilökohtaiset käyttäjätunnukset ja salasanat.

\subsection*{9. Tietojen tarkastusoikeus}
Henkilötietolain 26 § mukaan jäsenellä on oikeus tarkastaa itseään koskevat rekisteritiedot ja saada niistä pyydettäessä kopiot. Tarkastaminen on maksutonta kerran vuodessa.

\pagestyle{empty}

\internalappendix{2}{TKO-Äly ry:n jäsenrekisterin tietosuojaseloste}

\subsection*{Jäsenrekisterin tietosuojaseloste}
Tämä on EU:n yleisen tietosuoja-asetuksen mukainen rekisteri- ja tietosuojaseloste.
Laatimispäivämäärä 21.5.2018. Viimeisin muutos 7.12.2018
\subsection*{1. Rekisterinpitäjä}

TKO-äly ry
PL 68
00014 Helsinki
\subsection*{2. Yhteyshenkilö rekisteriä koskevissa asioissa}

TKO-äly ry:n hallitus
hallitus@tko-aly.fi
\subsection*{3. Rekisterin nimi}

TKO-äly ry:n jäsenrekisteri
\subsection*{4. Oikeusperuste ja henkilötietojen käsittelyn tarkoitus}

EU:n yleisen tietosuoja-asetuksen mukainen oikeusperuste henkilötietojen käsittelylle ovat
rekisterinpitäjän oikeutettu etu ja laillinen velvoite.
Henkilötietojen käsittelyn tarkoitus on yhdistyslain (503/1989) 11§ vaatima jäsenrekisterin
ylläpito, sekä ylläpitää yhdistyksen jäsenten yhteys- ja jäsenyystietoja.

\subsection*{5. Rekisterin tietosisältö}
Täydellinen nimi
Kotikunta
Sähköpostiosoite
Puhelinnumero
Helsingin yliopiston ylioppilaskunnan jäsenyys
Jäsenmaksuun ja jäsenyyteen liittyvät tiedot

\subsection*{6. Säännönmukaiset tietolähteet}
Jäsenhakemuksissa kerätyt yhteystiedot sekä Helsingin yliopiston tietojärjestelmästä tarkistettu opiskelijastatus.

\subsection*{7. Tietojen säännönmukaiset luovutukset}

Soveltuvin osin tietoja voidaan luovuttaa Helsingin Yliopiston Ylioppilaskunnan käyttöön opiskelijakunnan jäsenyyden tarkastamista varten. Muutoin tietoja ei luovuteta kolmansille osapuolille.
\subsection*{8. Tietojen siirto EU:n tai ETA:n ulkopuolelle}

Tietoja voidaan käsitellä Googlen pilvipalveluissa, jolloin käsiteltävät tiedot voivat sijaita EU:n tai ETA:n ulkopuolella. Google on sitoutunut noudattamaan pilvipalvelujensa osalta EU:n yleistä tietosuoja-asetusta ja Privacy Shield viitekehystä.

Edellä mainitun lisäksi tietoja ei siirretä Euroopan unionin tai Euroopan talousalueen ulkopuolelle.

\subsection*{9. Rekisterin suojauksen periaatteet}

Tietojen pääasiallinen säilöntäpaikka on yhdistyksen palvelimella sijaitseva tietokanta. Palvelin on suojattu ulkopuoliselta käytöltä ja jäsentietojen käyttöä valvotaan. Pääsy tietoihin on yhdistyksen tietojärjestelmien pääkäyttäjällä, rajatulla joukolla muita tämän valtuuttamia vastuuhenkilöitä ja yhdistyksen jäsenvastaavalla.

Väliaikaisesti tietoja voidaan käsitellä myös Googlen pilvipalveluissa, jolloin tietoihin on pääsy yhdistyksen jäsenvastaavalla tai laajimmillaan yhdistyksen hallituksella.

Molemmissa edellämainituissa tapauksissa tietojen käsittely edellyttää henkilökohtaista tunnistautumista.

\subsection*{10. Tarkastusoikeus}

Jokaisella rekisteriin kuuluvalla henkilöllä on oikeus tarkistaa rekisteriin hänestä tallennetut tiedot. Tietojen tarkistuspyyntö tulee lähettää kirjallisesti rekisterinpitäjälle. Rekisterinpitäjällä on tarvittaessa oikeus pyytää pyynnön esittäjää todistamaan henkilöllisyytensä. Rekisterinpitäjä vastaa pyynnön esittäjälle EU:n tietosuoja -asetuksessa säädetyssä ajassa (pääsääntöisesti kuukauden kuluessa). Tarkastaminen on maksutonta kerran vuodessa.

\subsection*{11. Oikeus vaatia tiedon korjaamista}

Jokaisella rekisteriin kuuluvalla henkilöllä on oikeus vaatia rekisteriin hänestä talletettujen tietojen korjausta. Tietojen korjauspyyntö tulee lähettää kirjallisesti rekisterinpitäjälle. Rekisterinpitäjällä on tarvittaessa oikeus pyytää pyynnön esittäjää todistamaan henkilöllisyytensä. Rekisterinpitäjä toteuttaa esittäjän pyynnön EU:n tietosuoja-asetuksessa säädetyssä ajassa (pääsääntöisesti kuukauden kuluessa).

\subsection*{12. Muut henkilötietojen käsittelyyn liittyvät tiedot}

Tietoja säilytetään kunnes rekisteröidyn jäsenyys yhdistyksessä päättyy. Jäsenyyden loputtua rekisteröidyn henkilötiedot poistetaan jäsenyyden loppumishetkellä menneillään olevan kalenterivuoden loppuun mennessä.

\subsection*{Poikkeukset:}

Mikäli rekisteröity osallistuu yhdistyksen tai sen hallituksen kokoukseen, kirjataan tämän nimi kokouspöytäkirjaan hyvän pöytäkirjatavan mukaisesti. Yhdistyksiä koskevan kirjanpitovelvoitteen vuoksi kokouspöytäkirjoja säilytetään 10 vuoden ajan.

Yhdistyksen talouden- ja kirjanpitovelvoitteista on määrätty muun muassa kirjanpito- ja yhdistyslaeissa. Täten henkilötiedot, esimerkiksi kirjanpidossa, tositteissa tai muissa vastaavissa talouden ja kirjanpidon asiakirjoissa, säilötään edellä mainittujen lakien mukaisesti vähintään 10 vuoden ajan.



\pagestyle{empty}

\internalappendix{3}{TKO-Äly ry:n tapahtumarekisterin tietosuojaseloste}

\subsection*{Tapahtumarekisterin tietosuojaseloste}
Tämä on EU:n yleisen tietosuoja-asetuksen mukainen rekisteri- ja tietosuojaseloste.
Laatimispäivämäärä 21.5.2018. Viimeisin muutos 24.5.2018

\subsection*{1. Rekisterinpitäjä}
TKO-äly ry
PL 68
00014 Helsinki
\subsection*{2. Yhteyshenkilö rekisteriä koskevissa asioissa}

TKO-äly ry:n hallitus
hallitus@tko-aly.fi

\subsection*{3. Rekisterin nimi}

TKO-äly ry:n tapahtumarekisteri

\subsection*{4. Oikeusperuste ja henkilötietojen käsittelyn tarkoitus}

EU:n yleisen tietosuoja-asetuksen mukainen oikeusperuste henkilötietojen käsittelylle on
rekisterinpitäjän oikeutettu etu.

Henkilötietojen käsittelyn tarkoitus on tapahtumien organisointiin ja niiden maksuliikenteeseen liittyvien toimien mahdollistaminen.

\subsection*{5. Rekisterin tietosisältö}
Koko nimi
Sähköpostiosoite
Puhelinnumero
*Tapahtuman maksutieto
*Erityisruokavalio
*Juomamieltymys
*Mahdolliset tapahtumakohtaiset lisätietokentät
(Tähdellä merkittyjä kenttiä käytetään vain tarvittaessa)

\subsection*{6. Säännönmukaiset tietolähteet}

Tapahtumailmoittautumisen yhteydessä kerätyt tiedot.

\subsection*{7. Tietojen säännönmukaiset luovutukset}

Tapahtumailmoittautumisen yhteydessä ilmoittautujalla on vapaaehtoinen mahdollisuus antaa lupa nimen julkaisuun, jolloin nimi julkaistaan tapahtumaan ilmoittautuneiden listalla.

Edellä mainitun lisäksi tietoja ei julkaista tai luovuteta muille tahoille.

\subsection*{8. Tietojen siirto EU:n tai ETA:n ulkopuolelle}

Tietoja voidaan käsitellä Googlen pilvipalveluissa, jolloin käsiteltävät tiedot voivat sijaita EU:n tai ETA:n ulkopuolella. Google on sitoutunut noudattamaan pilvipalvelujensa osalta EU:n yleistä tietosuoja-asetusta ja Privacy Shield viitekehystä.

Edellä mainitun lisäksi tietoja ei siirretä Euroopan unionin tai Euroopan talousalueen ulkopuolelle.

\subsection*{9. Rekisterin suojauksen periaatteet}

Tietojen pääasiallinen säilöntäpaikka on yhdistyksen palvelimella sijaitseva tietokanta. Palvelin on suojattu ulkopuoliselta käytöltä ja jäsentietojen käyttöä valvotaan. Pääsy tietoihin on yhdistyksen tietojärjestelmien pääkäyttäjällä, rajatulla joukolla muita tämän valtuuttamia vastuuhenkilöitä ja yhdistyksen tapahtumista vastuussa olevilla henkilöillä.

Väliaikaisesti tietoja voidaan käsitellä myös Googlen pilvipalveluissa, jolloin tietoihin on pääsy yhdistyksen jäsenvastaavalla tai laajimmillaan yhdistyksen hallituksella.

Molemmissa edellämainituissa tapauksissa tietojen käsittely edellyttää henkilökohtaista tunnistautumista.

\subsection*{10. Tarkastusoikeus}

Jokaisella rekisteriin kuuluvalla henkilöllä on oikeus tarkistaa rekisteriin hänestä tallennetut tiedot. Tietojen tarkistuspyyntö tulee lähettää kirjallisesti rekisterinpitäjälle. Rekisterinpitäjällä on tarvittaessa oikeus pyytää pyynnön esittäjää todistamaan henkilöllisyytensä. Rekisterinpitäjä vastaa pyynnön esittäjälle EU:n tietosuoja-asetuksessa säädetyssä ajassa (pääsääntöisesti kuukauden kuluessa). Tarkastaminen on maksutonta kerran vuodessa.

\subsection*{11. Oikeus vaatia tiedon korjaamista}

Jokaisella rekisteriin kuuluvalla henkilöllä on oikeus vaatia rekisteriin hänestä talletettujen tietojen korjausta. Tietojen korjauspyyntö tulee lähettää kirjallisesti rekisterinpitäjälle. Rekisterinpitäjällä on tarvittaessa oikeus pyytää pyynnön esittäjää todistamaan henkilöllisyytensä. Rekisterinpitäjä toteuttaa esittäjän pyynnön EU:n tietosuoja-asetuksessa säädetyssä ajassa (pääsääntöisesti kuukauden kuluessa).

\subsection*{12. Muut henkilötietojen käsittelyyn liittyvät tiedot}

Tietoja säilytetään tapahtuman järjestämisajankohdalla menneillään olevan tilikauden tilinpäätöksen vahvistamiseen saakka, kuitenkin enintään kuluvaa kalenterivuotta seuraavan kalenterivuoden huhtikuun loppuun saakka.

\subsection*{Poikkeus:}

Yhdistyksen talouden- ja kirjanpitovelvoitteista on määrätty muun muassa kirjanpito- ja yhdistyslaeissa. Täten henkilötiedot, esimerkiksi kirjanpidossa, tositteissa tai muissa vastaavissa talouden ja kirjanpidon asiakirjoissa, säilötään edellä mainittujen lakien mukaisesti vähintään 10 vuoden ajan.


\pagestyle{empty}

\internalappendix{4}{TKO-Äly ry:n tietosuojapolitiikka}

\subsection*{Tietosuojapolitiikka}
Tämän dokumentin tarkoitus on välittää TKO-äly ry:n tietosuojapolitiikka niille henkilöille, jotka käsittelevät yhdistyksen käytössä olevia henkilörekistereitä. Laatimispäivämäärä 21.5.2018. Viimeisin muutos 27.8.2018

Yhdistyksen käytössä on tällä hetkellä seuraavat rekisterit:

\begin{itemize}
\item Jäsenrekisteri  
\item Tapahtumien ilmoitusrekisteri
\item Kurssikarttasovelluksen käyttäjärekisteri
-\item Fuksipassin käyttäjärekisteri
\end{itemize}
Tämä dokumentti sekä mainittujen rekisterien tietosuojaselosteet tulee päivittää aina kun se on tarpeellista. Edellämainitut tulee tarkistaa ja tarvittaessa päivittää kalenterivuosittain yhdistyksen hallituksen sekä virkailijoiden vaihduttua, viimeistään tammikuun viimeisenä päivänä. Vastuu tietosuojadokumentaation tarkastamisesta ja päivittämisestä on kollektiivisesti yhdistyksen sen hetkisellä hallituksella.

Jokaisen yhdistyksen hallituksen jäsenen ja henkilötietoja käsittelevän virkailijan velvoitetaan tutustuvan ajantasaiseen tietosuojapolitiikkaan ja henkilötietorekistereitä koskeviin tietosuojaselosteisiin.

Yleisesti henkilötietojen käsittelyssä tulee noudattaa pienimmän haitan periaatetta, jonka kohteena on rekisteröity henkilö.


\subsection*{Käyttöoikeudet}

Henkilötietoja sisältävien rekisterien käyttöoikeudet tulee rajata vain niille henkilöille, joille se on välttämätöntä. Käyttöoikeudet tulee päivittää aina kun se on tarpeellista ja tarkistaa kalenterivuosittain, viimeistään tammikuun viimeisenä päivänä. Vastuu käyttöoikeuksien päivittämisestä on yhdistyksen Senior Application Evangelist virkaa hoitavalla vastuuhenkilöllä siltä osin kuin se on mahdollista. Viimekädessä vastuu on kuitenkin kollektiivisesti yhdistyksen senhetkisellä hallituksella.


\subsection*{Henkilötietojen elinkaari}

Henkilötietoja tulee säilyttää vain niin kauan kuin se palvelee alkuperäistä keräystarkoitusta. Tietojen enimmäissäilytysaika tulee määrittää yksityiskohtaisesti jokaisen rekisterin tietosuojaselosteessa.


\subsection*{Tietosuojan tekninen toteutus}

Henkilötietojen suojaamiseen ja säilyvyyteen tulee kiinnittää erityistä huomiota. Käyttöoikeuksien hallinnan ohella käytettyjen ratkaisujen yleinen tietoturva on merkittävässä roolissa. Kaikki tekniset toteutukset tulee toteuttaa parhaan mahdollisen kyseisellä hetkellä vallitsevan tietotaidon ja osaamisen puitteissa. Erityisesti järjestelmien rakentamisessa tulee kiinnittää huomiota EU:n yleisen tietosuoja-asetuksen artiklassa 25 määriteltyyn sisäänrakennetun ja oletusarvoisen tietosuojan periaatteeseen.


\subsection*{Henkilötietojen tarkistus ja korjaus}

Jokaisella rekisteriin kuuluvalla henkilöllä on oikeus tarkistaa rekisteriin hänestä tallennetut tiedot ja pyytää niiden korjausta. Tietojen tarkistus- ja korjauspyynnöt tulee lähettää kirjallisesti rekisterinpitäjälle. Rekisterinpitäjällä on tarvittaessa oikeus pyytää pyynnön esittäjää todistamaan henkilöllisyytensä. Rekisterinpitäjä vastaa rekisteröidyn pyyntöön tai suorittaa rekisteröidyn pyytämän korjauksen EU:n tietosuoja-asetuksessa säädetyssä ajassa (pääsääntöisesti kuukauden kuluessa). Mikäli pyyntöä ei voida toteuttaa kuukauden kuluessa on tästä ilmoitettava rekisteröidylle, jolloin voidaan soveltaa korkeintaan kahden kuukauden lisäaikaa.

Henkilötiedot voi tarkastaa veloituksetta kerran kalenterivuodessa. Myöhemmistä tarkastuspyynnöistä veloitetaan 60€ suuruinen palvelumaksu. Henkilötietojen korjauksesta ei veloiteta koskaan, sillä EU:n yleisen tietosuoja-asetuksen nojalla rekisterinpitäjällä on velvollisuus varmentaa henkilötietojen paikkansapitävyys.


\subsection*{Tietojen julkaisu}

Mikäli halutaan julkaista henkilötietoja, tulee siitä kerätä rekisteröidyltä aina erillinen suostumus ja kyseisen suostumuksen tulee aina olla aidosti vapaaehtoinen. Mahdollisuus henkilötietojen julkaisuun tulee kirjata kyseisen rekisterin tietosuojaselosteeseen. Esimerkiksi tapahtumailmoituksen tapauksessa ilmoittauduttaessa rekisteröidyltä kysytään lupa nimen julkaisemiseen tapahtuman osallistujalistassa. Vaikka lupaa ei annettaisikaan, tapahtumaan voi silti osallistua.

Poikkeuksen edelliseen muodostavat yhdistyksen ja sen hallituksen kokoukset, joihin osallistuvien nimet on kirjattava pöytäkirjaan. Tämän vuoksi kokouskutsuihin tulee kirjata maininta kokoukseen osallistuvien nimen kirjaamisesta kokouksen pöytäkirjaan. Yhdistyksen julkaisemien epävirallisten pöytäkirjojen osalta henkilötiedon julkaisuun tulee kysyä lupa. Julkaisusta kieltäytyneiden nimet voidaan kirjata julkisiin epävirallisiin pöytäkirjoihin anonyyminä tilastotietona.

\internalappendix{5}{TEK:n tietosuojaselostemalli}


\subsection*{Tietosuojaseloste – jäsenrekisteri}

Tämä on EU:n yleisen tietosuoja-asetuksen (GDPR) mukainen rekisteri- ja tietosuojaseloste. 
Laatimispäivämäärä 27.3.2018. Viimeisin muutos 25.5.2018.

\subsection*{1.	Rekisterinpitäjä}
Akateeminen Karkkikerho (AKK)
Jämeräntaival 7 M 2, 02150 Espoo 

\subsection*{2.	Rekisteristä vastaava yhteyshenkilö} 
Sihteeri Anssi Esimerkki
Sihteeri (at) karkkikerho.fi

\subsection*{3.	Rekisterin nimi}
Akateemisen Karkkikerhon jäsenrekisteri

\subsection*{4.	Oikeusperuste ja henkilötietojen käsittelyn tarkoitus}
EU:n yleisen tietosuoja-asetuksen mukainen oikeusperuste henkilötietojen käsittelylle on rekisterinpitäjän oikeutettu etu. 
Henkilötietojen käsittelyn tarkoitus on yhdistyslain (503/1989) 11§ vaatima jäsenrekisterin ylläpito, sekä ylläpitää kerhon jäsenien yhteystietoja.

\subsection*{5.	Rekisterin tietosisältö}
Rekisteriin tallennetaan seuraavia tietoja:

\begin{itemize}
\item Täydellinen nimi
\item Kotipaikkakunta
\item AYY:n jäsenyys
\item Sähköpostiosoite
\item Sähköpostilistoihin kuuluminen
\end{itemize}
Rekisterissä ei säilytetä kerhoon kuulumattomien henkilöiden tietoja, vaan nämä poistetaan välittömästi mikäli jäsenyys puretaan.

\subsection*{6.	Säännönmukaiset tietolähteet}
Rekisteriin tallennettavat tiedot saadaan kerhoon liittyviltä ja jo liittyneiltä jäseniltä www-lomakkeilla sekä kirjallisesti tapahtumissa.

\subsection*{7.	Tietojen luovuttaminen ja tietojen siirto EU:n tai ETA:n ulkopuolelle}
Tietoja ei luovuteta muille tahoille. Tietoja ei luovuteta, eikä säilytetä Euroopan Unionin tai Euroopan talousalueen ulkopuolella.

\subsection*{8.	Rekisterin suojauksen periaatteet}
Rekisterin käsittelyssä noudatetaan huolellisuutta. Tietoja säilytetään vain sähköisessä muodossa luotettavassa ja EU:n tietosuojalainsäädäntöä noudattavan kolmannen osapuolen internet-palvelimilla. Rekisterin salaus toteutetaan AKK:n omilla salausavaimilla. Tiedot eivät ole luettavissa muiden kuin AKK:n vastuuhenkilöiden toimesta.
Rekisterinpitäjä huolehtii siitä, että rekisteriin on pääsy vain asiaankuuluvilla henkilöillä.

\subsection*{9.	Tarkastusoikeus ja oikeus vaatia tiedon korjaamista }
Jokaisella rekisterissä olevalla henkilöllä on oikeus tarkistaa rekisteriin tallennetut tietonsa ja vaatia mahdollisen virheellisen tiedon korjaamista tai puutteellisen tiedon täydentämistä. Mikäli henkilö haluaa tarkistaa hänestä tallennetut tiedot tai vaatia niihin oikaisua, pyyntö tulee lähettää kirjallisesti rekisterinpitäjälle. Rekisterinpitäjä voi pyytää tarvittaessa pyynnön esittäjää todistamaan henkilöllisyytensä. Rekisterinpitäjä vastaa asiakkaalle EU:n tietosuoja-asetuksessa säädetyssä ajassa (pääsääntöisesti kuukauden kuluessa).

\subsection*{10.	Rekisterin tietojen säilytysaika}
Rekisterissä säilytetään vain niiden henkilöiden tietoja, jotka ovat AKK:n jäseniä. Kerhosta eronneiden tai erotettujen henkilöiden tiedot poistetaan rekisteristä kohtuullisen ajan kuluessa viimeistään kuukauden päästä eroamis- tai erottamispäätöksestä.




\internalappendix{6}{TEK:n tietosuojapolitiikkamalli}

\subsection*{Akateemisen Karkkikerhon tietosuojapolitiikka}
Tämän dokumentin tarkoitus on välittää tietoa Akateemisen Karkkikerhon(AKK) tietosuojan periaatteet niille kerhon henkilöille, jotka ylläpitävät tai käsittelevät kerhon hallussa olevia henkilörekistereitä. 
Kerhon hallussa on tällä hetkellä seuraavat rekisterit:
\begin{itemize}
    \item Jäsenrekisteri
    \item Tapahtumien ilmottautumisrekisteri
\end{itemize}
Tämä dokumentti sekä mainittujen rekisterien rekisteriselosteet tulee päivittää vuosittain aina hallituksen sekä toimihenkilöiden vaihduttua, viimeistään tammikuun 31. päivänä. Vastuu päivityksestä on nimetyllä rekisterivastaavalla tai tämän puuttuessa sihteerillä.  

\subsection*{Tietosuojarekisterien käyttöoikeudet}
Käyttöoikeuksien rajoittaminen on merkittävässä roolissa tietosuojaongelmien estämisessä. Oikeudet tulisi rajoittaa vain niihin henkilöihin, joilla on kerhon toimien kannalta siihen tarvetta. Esimerkiksi koko hallitukselle tai kaikille toimihenkilöille ei tunnuksia kannata jakaa heti toimikauden alussa, vaan aluksi vain rekisterin vastuu/yhteyshenkilölle. Kerhomme vastuuhenkilö on ilman erillistä päätöstä aina virkaa tekevä sihteeri.
Vastaavasti vanhalta hallitukselta ja toimihenkilöitä tulisi poistaa oikeudet jo heti kauden vaihduttua, ellei ole selvää, että kyseinen henkilö tulee tarvitsemaan pääsyä rekisteriin uuden kauden toimivirassaan. 
Vanhojen tunnusten poisto tulee tehdä viimeistään tammikuun 31. päivänä. Vastuu päivityksestä on nimetyllä rekisterivastaavalla tai tämän puuttuessa sihteerillä.  

\subsection*{Tietojen elinkaari}
Rekistereissä olevaa tietoa tulee säilyttää vain niin kauan kun se palvelee tarkoitustaan. Jäsenrekisterissä tulee luonnollisesti olla kaikkien kerhon sen hetkisten jäsenten tiedot, ja poistaa tiedot sitä mukaa kun jäsen poistuu kerhosta. 
Tapahtumien ilmottautumisrekisterin kohdalla tiedot tulee poistaa siinä vaiheessa, kun niillä ei voida katsoa olevan enää tarpeellista sen tapahtuman osalta, mihin tiedot on kerätty. Normaalien sitsien osalta tämä tarkoittaa noin kahta viikkoa, jolloin mahdolliset jälkipyykit ovat saatu hoidettua. 
Tapahtumiin osallistuneista voidaan kuitenkin kerätä nk. anonyymidataa, eli osallistujien lukumääriä, liha/kasvisruokailevien määriä ja muita tietoja jotka eivät ole linkitettyjä rekisterin dataan. Tälläiset tiedot tulee säilyttää erillisessä rekisterissä.

\subsection*{Tietosuojan tekninen varmistaminen}
Käyttöoikeuksien ohella rekisterien tekninen toteutus on tärkeää. Tällä hetkellä kerhon rekisterit säilytetään kerhon omalla koneella, missä pääsy tietoihin on rajoitettu salasanalla ja tietojen salaamisella.

Tapahtumien ilmottautumisrekisteriin tallennetaan tieto henkilön erikoisruokavaliosta. Sanamuoto tässä kohdin on tärkeä, sillä allergiatiedot ovat arkaluontoista henkilötietoa, kun erikoisruokavaliot voidaan katsoa normaaliksi henkilötiedoksi. Arkaluontoisen henkilötiedon tietosuojavaatimukset lainsäädännön osalta ovat huomattavasti tiukemmat. 

\subsection*{Henkilön oikeus pyytää ja korjata tietonsa}
Yksityishenkilöllä on oikeus pyytää AKK:n toimesta hänestä tallennetut henkilötiedot. Käytännössä tämä tarkoittaa jäsenrekisterin tietoja.
Koska tallennamme jäsenrekisteriin vain hyvin vähän tietoja henkilöistä, nopea haku sekä jäsenrekisteristä ja tapahtumarekisteristä ei ole vaikea toteuttaa. Tiedot tulee luovuttaa ensisijaisesti samassa formaatissa kuin pyyntö tuli, eli sähköpostiin tulee vastata sähköisellä tietojen luovuttamisella. Pyynnön kohtuullinen käsittelyaika on yksi kuukausi.
Tarvittaessa pyytäjää voidaan pyytää todistamaan henkilöllisyytensä.

Vastaavasti henkilöillä on oikeus pyytää tietojensa korjausta. Myös tietojen korjauksen kohtuullinen käsittelyaika on yksi kuukausi. 





\end{document}